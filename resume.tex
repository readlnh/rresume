% !TEX program = xelatex
% This is my resume
% by readlnh

\documentclass{resume}

\usepackage{lastpage}
\usepackage{fancyhdr}
\usepackage{linespacing_fix} % disable extra space before next section

\begin{document}
\pagestyle{fancy}
\fancyhf{}
\renewcommand\headrulewidth{0pt}
\cfoot{\thepage\ of \pageref{LastPage}}

\name{Xu LingXiao}

\basicInfo{
  \email{readlnh@hotmail.com} \textperiodcentered\
  \phone{(+86) 184-3436-1246} \textperiodcentered\
  \github[readlnh]{https://github.com/readlnh}
  % \linkedin[user]{https://www.linkedin.com/in/user}
}

\section{\faGraduationCap\ Education}
\datedsubsection{\textbf{North University of China (NUC)},  CN}{09/14 -- 06/18}
  Major: 
    Electronic and Information Engineering

\section{\faUsers\ Work Experience}
\datedsubsection{\textbf{Institute of Acoustics of the Chinese Academy of Sciences}}{03/18 -- 06/18}
\role{Intern}{Ultrasonic Photoelastic Visualization Development}
\begin{itemize}
  \item Designed the basic set-up for the photo-elastic system including the delay circuit for timing and delay of illumination with respect to the pulse.
  \item Developed the software of the whole system, including camera control,  image capture, image scaling, image processing.
  \item Learned a lot about C++ programming and Qt5.
\end{itemize}

\datedsubsection{\textbf{National University Of Defense Technology}, Full-time}{07/18 -- 06/19}
\role{Operating System Engineer}{Mobile OS Team}
\begin{itemize}
  \item Improved many libraries used by Kylin OS. For example,
    optimized the display performance of graphics inside,
    and added new features to the sensor framework.
  \item Ported the OS to different devices, such as ST700, Nexus 4, Nexus 7.
\end{itemize}

\role{Operating System Engineer, Desktop Environment Developer}{Ubuntu Kylin Team}
\begin{itemize}
  \item Maintained the open source projects of UbuntuKylin. Created the Devops System of the projects including automatic tools and testing frameworks.
  \item Developed and Maintained ukui desktop environment. Added HiDPI support to the desktop enviroment, upgraded the system tray, followed the rules of the freedesktop.
  \item Customized a special UbuntuKylin version which could running in the docker with full desktop environment, replaced some Dbus services and customized the graphics under the desktop environment.
  \item Created UbuntuKylin Wiki, based on nodejs.
\end{itemize}

\datedsubsection{\textbf{Itering Technology}, Full-time}{11/19 -- Now}
\role{Blockchain Engineer}{Rust developing groups}
\begin{itemize}
  \item Ported the blockchain to loogson mipsel board.
  \item Development the fundanmental of the darwinia network.
\end{itemize}

\section{\faGithubAlt \ Personal Projects}


\datedsubsection{\textbf{rtoyos}}{\url{https://github.com/readlnh/rtoyos}}
Simple,mini Operating System kernel written by C Programming Language
\begin{itemize}
  \item Using the Multiboot standard, bootloading with Grub2.
  \item Supporting simple memory management.
  \item Supporting multitasks.
\end{itemize}

\datedsubsection{\textbf{rdocker}}{\url{https://github.com/readlnh/rdocker}}
My own docker written by Golang(Developing...)
\begin{itemize}
  \item Isolating applications in a virtual private root but taking the process further.
  \item Supporting six Namespaces, including UTS(hostname), IPC(inter-process communication), NS(mount points), NET(network access), USER(map virtual, local user-ids to real local ones).
  \item Using Cgroups to conrol the resources. 
\end{itemize}

\section{\faHeartO\ Honors \& Awards}
\datedline{The ACM-ICPC Asia Regional Contest Hefei Site 2015  \textit{Bronze Medal}}{10/15}
\datedline{The 7th Contest of LAN QIAO CUP Software B group Shanxi \textit{First prize}}{03/16}
\datedline{The 7th Contest of LAN QIAO CUP Software B group Final \textit{First prize(top50)}}{05/16}
\datedline{The Second Group Programming Ladder Tournament Northwestern Regional Contest \textit{First Price}}{03/17}
\datedline{The Second Group Programming Ladder Tournament Northwestern Final Contest \textit{Second Price}}{07/17}


\section{\faCogs\ Skills}
\begin{itemize}[parsep=0.25ex]
  \item \textbf{Programming Languages}:
    \textbf{multilingual developer} (not limited to any specific language),
    and especially experienced in C/Rust/Python/Go,
    comfortable with C++/C\#/lisp/ruby (listed in random order).

  % Operating System theories
  \item \textbf{Operating System}:
    familiar with the graphics architecture of Android/Linux, including both low-level and high-level components
    understand MemoryBlock, IPC and Threading Models, studied some theories about filesystem.

  % language C/C++
  \item \textbf{C/C++}:
    \textbf{4 years} of experience,
    familiar with gtk+/Qt programming, Linux Programming, the CMake build tool, have experience
    in OpenCV/OpenGL.

  \item \textbf{Algorithm Theories}:
    familiar with Dymaic Programming, Greedy Algorithms, Data Structures, Graph Algorithms,
    understand Computational Geometry and Number theory.

  \item \textbf{Developing Tools}:
    can adapt to any editors/OSs, usually use vim/VsCode and JetBrains IDEs under Ubuntu.
    Experience working with Git and Git workflow.

  \item \textbf{Devops}:
    Experience working with docker and k8s.      

 \item \textbf{Languages}:
    \item Languages: English - fluent, Chinese - native speaker
\end{itemize}



\section{\faInfo\ Miscellaneous}
\begin{itemize}[parsep=0.25ex]
  \item Blog: \url{https://readlnh.github.io/} haven't translated into English
  \item Member of Chinese Rust Programming Community
  \item Ubuntu developer
  \item Books about operating system I've translated: \url{https://github.com/readlnh/Writing-an-OS-in-Rust-Second-Edition-zh_CN}
  \item Opensource contributions: 
    pull-requested (functional improvements) to organizations like \textit{Rust, Ubuntu, Redox}
  \item Love making friends
\end{itemize}

%% Reference
%\newpage
%\bibliographystyle{IEEETran}
%\bibliography{mycite}
\end{document}

