% !TEX program = xelatex
% This is my resume
% Chinese translation
% by ice1000

\documentclass{resume}

\usepackage{lastpage}
\usepackage{fancyhdr}
\usepackage{linespacing_fix} % disable extra space before next section
\usepackage[fallback]{xeCJK}

%% \setmainfont[]{SimSun}
%% \setCJKfallbackfamilyfont{rm}{HAN NOM B}
%%\setCJKmainfont[BoldFont=Sarasa Gothic SC,ItalicFont=KaiTi_GB2312]{Source Han Serif SC}
%% \renewcommand{\thepage}{\Chinese{page}}

\begin{document}
\pagestyle{fancy}
\fancyhf{}
\renewcommand\headrulewidth{0pt}
\cfoot{\thepage\ of \pageref{LastPage}}

\name{徐凌霄}

\basicInfo{
  \email{readlnh@163.com} \textperiodcentered\ 
  \phone{(+86) 184-3436-1246} \textperiodcentered\ 
  \github[readlnh]{https://github.com/readlnh}
  % \linkedin[user]{https://www.linkedin.com/in/user}
}

\section{\faGraduationCap\ 教育经历}
\datedsubsection{\textbf{中北大学}, 中国}{2014.9 -- 2018.7}
  专业:电子信息工程

\section{\faUsers\ 工作/实习经历}
\datedsubsection{\textbf{中国科学院声学所}, 北京, 中国}{2018.2 -- 2018.6}
\role{实习}{超声光弹系统的上位机开发}
\begin{itemize}
  \item 负责超声光弹系统的上位机开发,采用qt5相关技术
  \item 修复底层电路问题,并通过单片机编程实现激光和超声信号的同步控制
  \item 通过上位机实现对旋转镜架和相机的同步控制,以达到同步拍摄照片的效果
\end{itemize}

\datedsubsection{\textbf{国防科技大学银河技术服务部}, 长沙, 中国}{2018.8 -- 至今}
\role{全职}{操作系统研发工程师}
\begin{itemize}
  \item 针对终端os(基于firefoxos)进行开发和优化,主要负责图形和传感器两大块
    \begin{itemize}
      \item 适配了ST700,nexus4,nexus7(flo),nexus7(deb)等多台设备(其余涉密不列)
      \item 将终端os底层HAL从安卓4.4升级到安卓6.0,适配安卓6.0设备
      \item 调研b2g传感器框架,给出了终端os传感器移植方案,并用于多个军方项目,得到良好效果
      \item 针对某些特殊设备进行图形优化,解决裂屏现象  
    \end{itemize} 
  
  \item 进行ukui开源社区的维护,实现自动编包平台
  \begin{itemize}
    \item 实现了自动编辑changelog,修改维护者签名,以及版本号自增功能
    \item 实现了自动安装依赖,检查,并编包上传launchpad平台功能
    \item 实现了定时自启动任务同步github代码以达到每日更新软件源的效果
  \end{itemize}

  \item 进行ukui桌面环境的开发和维护,主要负责ukui-panel,ukui-indicator
    \begin{itemize}
      \item 修复panel的高分屏显示
      \item 优化了ukui-indicator的图标显示,解决了panel切换方向后图标位移的问题
    \end{itemize}
\end{itemize} 

\section{\faGithubAlt\ 个人项目}

\datedsubsection{\textbf{rtoyos}}{\url{https://github.com/readlnh/rtoyos}}
mini操作系统内核
\begin{itemize}
  \item 使用multiboot规范,采用grub2引导启动
  \item 实现了全局描述符表,中断描述符表,可自定义中断
  \item 实现了基本的内存管理
\end{itemize}

\datedsubsection{\textbf{rdocker}}{\url{https://github.com/readlnh/rdocker.git}}
mydocker在linux 4.18.0-17-generic版本内核上的实现(完善中)
\begin{itemize}
  \item 已实现linux namespace的隔离
  \item 已实现cgroups资源限制
\end{itemize}

\section{\faHeartO\ HONORS & AWARDS}
\datedline{ACM-ICPC 亚洲区合肥站 铜牌}{2015.10}
\datedline{第七届蓝桥杯大赛山西赛区 B 组 一等奖}{2016.03}
\datedline{第七届蓝桥杯大赛全国总决赛 B 组 一等奖}{2016.05}
\datedline{第二届中国高校计算机大赛西北赛区 团队一等奖}{2017.03}
\datedline{第二届中国高校计算机大赛总决赛 高校二等奖}{2017.07}

\section{\faCogs\ 技能}
\begin{itemize}[parsep=0.25ex]
  \item \textbf{编程语言}:
    工作主要使用c/c++开发,熟悉gtk+,glib,对golang/python/rust比较了解

  % compiler theories
  \item \textbf{操作系统}:
    \begin{itemize}
      \item  对操作系统的基本原理较为熟悉,熟练使用linux环境开发
      \item  熟悉安卓底层结构,了解安卓显示机制
    \end{itemize}

  \item \textbf{IDE 开发工具}:
    熟悉vim,shell,熟练使用linux平台下的各种工具
    熟悉git,以及git协作开发,有使用github,launchpad等平台的相关经验
    熟悉ubuntu平台编包规则和流程

\end{itemize}

% \section{\faHeartO\ Honors and Awards}
% \datedline{\textit{\nth{1} Prize}, Award on xxx }{Jun. 2013}
% \datedline{Other awards}{2015}

\section{\faInfo\ 其他}
\begin{itemize}[parsep=0.25ex]
  \item 博客: \url{https://blog.csdn.net/readlnh}
  \item githubpages 博客: \url{https://readlnh.github.io/}
  \item 知乎专栏: \url{https://zhuanlan.zhihu.com/c_1066709704398532608}
  \item 有技术热情,善于学习新知识
\end{itemize}

%% Reference
%\newpage
%\bibliographystyle{IEEETran}
%\bibliography{mycite}
\end{document}
